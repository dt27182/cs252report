\section{Introduction}
The primary way in which hardware designers create digital hardware design today is by specifying their design in Register Transfer Level (RTL) specification languages such as Verilog and VHDL. RTL specifications is a fairly low level specificaiton in which designers specify the cycle by cycle update of state elements within the digital design. Although RTL specification frees the designer from specifying every transistor or logic gate and their connections, RTL specification is still quite tedious to write. Additionally, in RTL specification, designers conflate the functionality of the digital circuit with hardware optimizations techniques such as pipelining, multi-threading, time-multiplexing, etc. This not only makes it difficult for the designer to make sure that a well optimized design is functionally correct, but it also makes it very difficult for the designer to explore the design space of optimization techniques.

In this paper, I propose a system in which the designer specifies in a RTL like Hardware Description Language(HDL) a minimally complex circuit that is functionally correct , but has no optimizations applied to it. The designer then separately annotates the minimal circuit with desired optimizations and uses a software tool to automatically apply the optimizations to the minimal circuit. The tool is capable of creating multi-thread in-order designs of any number of threads and any number of pipeline stages that is functionally equivalent to n-copies of the original minimal circuit, where n is the number of threads.

